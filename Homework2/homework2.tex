\documentclass[12pt]{article}
\usepackage[margin=1in]{geometry}
\usepackage{amsmath,amsthm,amssymb}

\newcommand{\N}{\mathbb{N}}
\newcommand{\Z}{\mathbb{Z}}

\title{MAE 275 - Homework 2}
\author{John Karasinski}

\begin{document}
\maketitle

\section{Problem 1}
We can define the longitudinal and lateral linearized aircraft equations of motion. The longitudinal equations can be expressed as

\begin{equation*}
\begin{split}
\Delta \dot{u} = &X_u \Delta u + X_w \Delta w - g\cos \theta_0 \Delta \theta \\
\Delta \dot{w} = &\dfrac{Z_u}{1-Z_{\dot{w}}} \Delta u +
                  \dfrac{Z_w}{1-Z_{\dot{w}}} \Delta w +
                  \dfrac{Z_q + u_0}{1-Z_{\dot{w}}} \Delta q -
                  \dfrac{g\sin \theta_0}{1-Z_{\dot{w}}} \Delta \theta \\
\Delta \dot{q} = &\left[ M_u + \dfrac{M_{\dot{w}} Z_u}{1-Z_{\dot{w}}} \Delta u  \right] +
                  \left[ M_w + \dfrac{M_{\dot{w}} Z_w}{1-Z_{\dot{w}}} \Delta w  \right] +
                  \left[ M_q + \dfrac{M_{\dot{w}} (Z_q + u_0)}{1-Z_{\dot{w}}} \Delta q  \right] -\left[ \dfrac{M_{\dot{w}} g\sin \theta_0}{1-Z_{\dot{w}}} \Delta \theta  \right] \\
\Delta \dot{\theta} = &\Delta q \\
\Delta \dot{h} = &-\Delta w + u_0 \Delta \theta \\
\end{split}
\end{equation*}

\noindent or in state space form, with state variables $\Delta u, \Delta w, \Delta q, \Delta \theta, \Delta h $, as

\begin{equation*}
A =
\begin{bmatrix}
    X_u & X_w & 0 & -g \cos(\theta_0) & 0 \\
    \\
    \dfrac{Z_u}{1-Z_{\dot{w}}} & \dfrac{Z_w}{1-Z_{\dot{w}}} & \dfrac{Z_q + u_0}{1-Z_{\dot{w}}} & \dfrac{g\sin \theta_0}{1-Z_{\dot{w}}} & 0 \\
    \\
    M_u + \dfrac{M_{\dot{w}} Z_u}{1-Z_{\dot{w}}} & M_w + \dfrac{M_{\dot{w}} Z_w}{1-Z_{\dot{w}}} & M_q + \dfrac{M_{\dot{w}} (Z_q + u_0)}{1-Z_{\dot{w}}} & -\dfrac{M_{\dot{w}} g\sin \theta_0}{1-Z_{\dot{w}}} & 0 \\
    \\
    0 & 0 & 1 & 0 & 0 \\
    \\
    0 & -1 & 0 & u_0 & 0
\end{bmatrix}\\
\end{equation*}

\noindent Plugging in the data for the F-89 aircraft (Flight Condition 8901) on pages A3-A5 in the Appendix of "Aircraft Dynamics and Automatic Control" yields
\begin{equation*}
A =
\begin{bmatrix}
    1 & 2 & 3 & 4 & 5 \\
    1 & 2 & 3 & 4 & 5 \\
    1 & 2 & 3 & 4 & 5 \\
    1 & 2 & 3 & 4 & 5 \\
    1 & 2 & 3 & 4 & 5 \\
\end{bmatrix}
\end{equation*}

\noindent The lateral equations can be expressed as
\begin{equation*}
\begin{split}
\Delta \dot{v} = & Y_v\Delta v + Y_p\Delta p + \left[ Y_r - u_0 \right]\Delta r + g\cos\theta_0\Delta \varphi \\
\Delta \dot{p} = & L_v^\prime \Delta v + L_p^\prime \Delta p + L_r^\prime \Delta r\\
\Delta \dot{r} = & N_v^\prime \Delta v + N_p^\prime \Delta p + N_r^\prime \Delta r\\
\Delta \dot{\varphi} = &\Delta p + r\tan\theta_0\Delta r\\
\Delta \dot{\psi} = &r\sec\theta_0\Delta r\\
\end{split}
\end{equation*}


\noindent or in state space form, with state variables $\Delta v, \Delta p, \Delta r, \Delta \varphi \mbox{(roll)}, \Delta \psi $, as

\begin{equation*}
A =
\begin{bmatrix}
    Y_v & Y_p & \left[ Y_r-u_0 \right] & g\cos \theta_0 & 0 \\
    \\
    L_v^\prime & L_p^\prime & L_r^\prime & 0 & 0 \\
    \\
    N_v^\prime & N_p^\prime & N_r^\prime & 0 & 0 \\
    \\
    0 & 1 & \tan \theta_0 & 0 & 0 \\
    \\
    0 & 0 & \sec \theta_0 & 0 & 0
\end{bmatrix}\\
\end{equation*}

\noindent Plugging in the data for the F-89 aircraft (Flight Condition 8901) on pages A3-A5 in the Appendix of "Aircraft Dynamics and Automatic Control" yields

\begin{equation*}
A =
\begin{bmatrix}
    1 & 2 & 3 & 4 & 5 \\
    1 & 2 & 3 & 4 & 5 \\
    1 & 2 & 3 & 4 & 5 \\
    1 & 2 & 3 & 4 & 5 \\
    1 & 2 & 3 & 4 & 5 \\
\end{bmatrix}
\end{equation*}

\section{Problem 2}


\section{Problem 3}
\section{Problem 4}

\end{document}
