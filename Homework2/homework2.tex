\documentclass[12pt]{article}
\usepackage[margin=1in]{geometry}
\usepackage{amsmath,amsthm,amssymb}

\usepackage[T1]{fontenc}
\usepackage{bigfoot} % to allow verbatim in footnote
\usepackage[numbered,framed]{matlab-prettifier}
\usepackage{filecontents}

\let\ph\mlplaceholder % shorter macro
\lstMakeShortInline"

\lstset{
  style              = Matlab-editor,
  basicstyle         = \mlttfamily,
  escapechar         = ",
  mlshowsectionrules = true,
}

\title{MAE 275 - Homework 2}
\author{John Karasinski}

\begin{document}
\maketitle

\section{Problem 1}
We can define the longitudinal and lateral linearized aircraft equations of motion. The longitudinal equations can be expressed as

\begin{equation*}
\begin{split}
\Delta \dot{u} = &X_u \Delta u + X_w \Delta w - g\cos \theta_0 \Delta \theta \\
\Delta \dot{w} = &\dfrac{Z_u}{1-Z_{\dot{w}}} \Delta u +
                  \dfrac{Z_w}{1-Z_{\dot{w}}} \Delta w +
                  \dfrac{Z_q + u_0}{1-Z_{\dot{w}}} \Delta q -
                  \dfrac{g\sin \theta_0}{1-Z_{\dot{w}}} \Delta \theta \\
\Delta \dot{q} = &\left[ M_u + \dfrac{M_{\dot{w}} Z_u}{1-Z_{\dot{w}}} \Delta u  \right] +
                  \left[ M_w + \dfrac{M_{\dot{w}} Z_w}{1-Z_{\dot{w}}} \Delta w  \right] +
                  \left[ M_q + \dfrac{M_{\dot{w}} (Z_q + u_0)}{1-Z_{\dot{w}}} \Delta q  \right] -\left[ \dfrac{M_{\dot{w}} g\sin \theta_0}{1-Z_{\dot{w}}} \Delta \theta  \right] \\
\Delta \dot{\theta} = &\Delta q \\
\Delta \dot{h} = &-\Delta w + u_0 \Delta \theta \\
\end{split}
\end{equation*}

\noindent or in state space form, with state variables $\Delta u, \Delta w, \Delta q, \Delta \theta, \Delta h $, as

\begin{equation*}
A =
\begin{bmatrix}
    X_u & X_w & 0 & -g \cos(\theta_0) & 0 \\
    \\
    \dfrac{Z_u}{1-Z_{\dot{w}}} & \dfrac{Z_w}{1-Z_{\dot{w}}} & \dfrac{Z_q + u_0}{1-Z_{\dot{w}}} & \dfrac{g\sin \theta_0}{1-Z_{\dot{w}}} & 0 \\
    \\
    M_u + \dfrac{M_{\dot{w}} Z_u}{1-Z_{\dot{w}}} & M_w + \dfrac{M_{\dot{w}} Z_w}{1-Z_{\dot{w}}} & M_q + \dfrac{M_{\dot{w}} (Z_q + u_0)}{1-Z_{\dot{w}}} & -\dfrac{M_{\dot{w}} g\sin \theta_0}{1-Z_{\dot{w}}} & 0 \\
    \\
    0 & 0 & 1 & 0 & 0 \\
    \\
    0 & -1 & 0 & u_0 & 0
\end{bmatrix}\\
\end{equation*}

\noindent Plugging in the data for the F-89 aircraft (Flight Condition 8901) on pages A3-A5 in the Appendix of "Aircraft Dynamics and Automatic Control" yields
\begin{equation*}
A =
\begin{bmatrix}
    1 & 2 & 3 & 4 & 5 \\
    1 & 2 & 3 & 4 & 5 \\
    1 & 2 & 3 & 4 & 5 \\
    1 & 2 & 3 & 4 & 5 \\
    1 & 2 & 3 & 4 & 5 \\
\end{bmatrix}
\end{equation*}

\noindent The lateral equations can be expressed as
\begin{equation*}
\begin{split}
\Delta \dot{v} = & Y_v\Delta v + Y_p\Delta p + \left[ Y_r - u_0 \right]\Delta r + g\cos\theta_0\Delta \varphi \\
\Delta \dot{p} = & L_v^\prime \Delta v + L_p^\prime \Delta p + L_r^\prime \Delta r\\
\Delta \dot{r} = & N_v^\prime \Delta v + N_p^\prime \Delta p + N_r^\prime \Delta r\\
\Delta \dot{\varphi} = &\Delta p + r\tan\theta_0\Delta r\\
\Delta \dot{\psi} = &r\sec\theta_0\Delta r\\
\end{split}
\end{equation*}


\noindent or in state space form, with state variables $\Delta v, \Delta p, \Delta r, \Delta \varphi \mbox{(roll)}, \Delta \psi $, as

\begin{equation*}
A =
\begin{bmatrix}
    Y_v & Y_p & \left[ Y_r-u_0 \right] & g\cos \theta_0 & 0 \\
    \\
    L_v^\prime & L_p^\prime & L_r^\prime & 0 & 0 \\
    \\
    N_v^\prime & N_p^\prime & N_r^\prime & 0 & 0 \\
    \\
    0 & 1 & \tan \theta_0 & 0 & 0 \\
    \\
    0 & 0 & \sec \theta_0 & 0 & 0
\end{bmatrix}\\
\end{equation*}

\noindent Plugging in the appropriate data

\begin{equation*}
A =
\begin{bmatrix}
    1 & 2 & 3 & 4 & 5 \\
    1 & 2 & 3 & 4 & 5 \\
    1 & 2 & 3 & 4 & 5 \\
    1 & 2 & 3 & 4 & 5 \\
    1 & 2 & 3 & 4 & 5 \\
\end{bmatrix}
\end{equation*}

\newpage
\section{Problem 2}
\subsection{Longitudinal}
\noindent The following MATLAB command is called to identify the characteristc roots and eigenvector elements
\begin{filecontents*}{code.m}
[v, d] = eig(A);
\end{filecontents*}
\lstinputlisting[]{code.m}

\noindent resulting in two complex pairs

\begin{equation*}
\begin{split}
d_1 = X \pm i Y \\
d_2 = X \pm i Y \\
\end{split}
\end{equation*}

\noindent and their associated eigenvectors

\begin{equation*}
\begin{split}
v_1 = [1, 2, 3, 4, 5] \\
v_2 = [1, 2, 3, 4, 5] \\
\end{split}
\end{equation*}

\noindent Before exciting these modes, the rest of the state space system must be defined. The longitudinal A matrix from above is used, along with B, C, and D matrices defined as
\begin{filecontents*}{code.m}
B = [0; 0; 0; 0; 0];

C = [[1, 0, 0, 0, 0];
     [0, 1/u_0, 0, 0, 0];
     [0, 0, 1, 0, 0];
     [0, 0, 0, 1, 0];
     [0, 0, 0, 0, 1]];

D = [0; 0; 0; 0; 0];
\end{filecontents*}
\lstinputlisting[]{code.m}

\noindent finally, the initial state can be defined and the initial command can be run by
\begin{filecontents*}{code.m}
i1 = real(v(:,1));
initial(A, B, C, D, i1, 5)
\end{filecontents*}
\lstinputlisting[]{code.m}

\noindent and results in the following figures

\subsection{Lateral}
\noindent The lateral eigenvalues are identified as

\begin{equation*}
\begin{split}
d_1 = X \pm i Y \\
\ldots \\
d_n = X \pm i Y \\
\end{split}
\end{equation*}

\noindent and their associated eigenvectors

\begin{equation*}
\begin{split}
v_1 = [1, 2, 3, 4, 5] \\
\ldots \\
v_n = [1, 2, 3, 4, 5] \\
\end{split}
\end{equation*}

\noindent Before exciting these modes, the rest of the state space system must be defined. The lateral A matrix from above is used, along with B, C, and D matrices defined as
\begin{filecontents*}{code.m}
B = [0; 0; 0; 0; 0];

C = [[1/u_0, 0, 0, 0, 0];
     [0, 1, 0, 0, 0];
     [0, 0, 1, 0, 0];
     [0, 0, 0, 1, 0];
     [0, 0, 0, 0, 1]];

D = [0; 0; 0; 0; 0];
\end{filecontents*}
\lstinputlisting[]{code.m}

\noindent Exciting each of these modes with the appropriate eigenvector results in

\newpage
\section{Problem 3}
\section{Problem 4}

\end{document}
