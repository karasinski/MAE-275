\documentclass[a4paper]{article}

\usepackage[english]{babel}
\usepackage[utf8x]{inputenc}
\usepackage{amsmath}
\usepackage{graphicx}
\usepackage[colorinlistoftodos]{todonotes}

\newcommand{\ihat}{\boldsymbol{\hat{\textbf{i}}}}
\newcommand{\jhat}{\boldsymbol{\hat{\textbf{j}}}}
\newcommand{\khat}{\boldsymbol{\hat{\textbf{k}}}}

\title{MAE 275 - Homework 1}
\author{John Karasinski}

\begin{document}
\maketitle


\section{Problem 1}
Beginning with the integral equation from the handwritten notes
\begin{equation*}
\overline{M}_0 + \overline{M}_{T_0} + \overline{M}_{{IR}_0} = 
\iiint\limits_{sys} \overline{r}^2 \overline{\dot{\Omega}} dm - 
\iiint\limits_{sys} (\overline{r} \cdot \overline{\dot{\Omega}}) \overline{r} dm -
\iiint\limits_{sys} (\overline{\dot{\Omega}} \cdot \overline{r}) (\overline{r} \times \overline{\dot{\Omega}} ) dm
\end{equation*}

\noindent and defining
\begin{equation*}
\begin{split}
\overline{r}            &= x \ihat + y \jhat + z \khat \\
\overline{\Omega}       &= P \ihat + Q \jhat + R \khat \\
\overline{\dot{\Omega}} &= \dot{P} \ihat + \dot{Q} \jhat + \dot{R} \khat
\end{split}
\end{equation*}

\noindent one can find the z components of the moment equation. Inserting these definitions and ignoring the x and y components
\begin{equation*}
\begin{split}
N \khat =  & + \dot{R} \khat \iiint\limits_{sys} (x^2 + y^2 + z^2) dm \\
           &            - \iiint\limits_{sys} x \dot{P} z \khat dm  
                        - \iiint\limits_{sys} y \dot{Q} z \khat dm  
                        - \iiint\limits_{sys} z \dot{R} z \khat dm \\
           &            + \iiint\limits_{sys} (P x + Q y + R z)(Q x - P y) \khat dm \\
\end{split}
\end{equation*}


\noindent Simplifying and distributing terms
\begin{equation*}
\begin{split}
N =  & +\dot{R} \iiint\limits_{sys} (x^2 + y^2 + z^2) dm
            - \dot{P} \iiint\limits_{sys} x z dm
            - \dot{Q} \iiint\limits_{sys} y z dm
            - \dot{R} \iiint\limits_{sys} z^2 dm \\
          & + \iiint\limits_{sys} (P Q x^2 - P^2 x y + Q^2 x y - P Q y^2 + R Q x z - P R y z) dm \\
\end{split}
\end{equation*}

\noindent Adding and subtracting $ P Q \iiint\limits_{sys} z^2 dm $ to the RHS
\begin{equation*}
\begin{split}
N =  & + \dot{R} \iiint\limits_{sys} (x^2 + y^2) dm
            - \dot{P} \iiint\limits_{sys} x z dm
            - \dot{Q} \iiint\limits_{sys} y z dm \\
     &      +     P Q \iiint\limits_{sys} (x^2 + z^2) dm 
            -     P^2 \iiint\limits_{sys} x y dm 
            +     Q^2 \iiint\limits_{sys} x y dm \\
     &      -     P Q \iiint\limits_{sys} (y^2 + z^2) dm 
            +     R Q \iiint\limits_{sys} x z dm 
            -     P R \iiint\limits_{sys} y z dm \\
\end{split}
\end{equation*}

\noindent Recognizing geometric terms
\begin{equation*}
\begin{split}
N = & +\dot{R} I_z
    - \dot{P} I_{x z}
    - \dot{Q} I_{y z} \\
    & + P Q I_{y} 
      - P^2 I_{x y} 
      + Q^2 I_{x y} \\
    & - P Q I_{x} 
      + R Q I_{x z} 
      - P R I_{y z} \\
\end{split}
\end{equation*}

\noindent Finally, assuming an XZ plane of symmetry ($I_{yz} = I_{xy} = 0$)
\begin{equation*}
N =     \dot{R} I_z
      - \dot{P} I_{x z}
      + P Q (I_{y} - I_{x}) 
      + Q R I_{x z} 
\end{equation*}

\section{Problem 2}
\noindent Given the following z-force equation
\begin{equation*}
Z = Z_T - Z_g = m (\dot{W} + PV - QU - g\cos(\theta_0)\cos(\phi_0))
\end{equation*}

\noindent it is possible to linearize this equation using a perturbation method. Adding a small disturbance to each term, so that
\begin{equation*}
\begin{alignedat}{2}
Z &= Z_0 & &+ dZ \\
\dot{W} &= \dot{W_0} & &+ \dot{w} \\
P &= P_0 & &+ p \\
V &= V_0 & &+ v \\
Q &= Q_0 & &+ q \\
U &= U_0 & &+ u
\end{alignedat}
\end{equation*}
\begin{equation} \label{eq:expansion}
\begin{split}
c(\theta_0 + \theta)c(\phi_0 + \phi) = & s\theta_0  s\phi_0 s\theta s\phi + c\theta_0 c\phi_0  c\theta c\phi - \\
                                       & c\theta_0  s\phi_0  c\theta s\phi - s\theta_0 c\phi_0 s\theta c\phi
\end{split}
\end{equation}

\noindent The disturbances from the steady flight conditions are assumed to be small enough so that the sines and cosines of the disturbance angles are approximately the angles themselves and 1, respectively, and so that the products and squares of the disturbance quantities are negligible compared to the quantities themselves. \\

\noindent Applying this to Equation~\ref{eq:expansion}, we have
\begin{equation*}
c(\theta_0 + \theta)c(\phi_0 + \phi) = c\theta_0 c\phi_0  - c\theta_0  s\phi_0 \phi - s\theta_0 c\phi_0 \theta
\end{equation*}

\noindent Plugging these terms into our z-force equation results in
\begin{equation*}
\begin{split}
Z_0 + dZ = & m [\dot{W_0} + \dot{w} + (P_0 + p) (V_0 + v) - (Q_0 + q) (U_0 + u) - \\ 
    & g (c\theta_0 c\phi_0  - c\theta_0  s\phi_0 \phi - s\theta_0 c\phi_0 \theta)]
\end{split}
\end{equation*}

\noindent Multiplying the terms out results in
\begin{equation*}
\begin{split}
Z_0 + dZ = m[ & \dot{W_0} + \dot{w} + P_0 V_0 + P_0 v + V_0 p + p v \\
       & - Q_0 U_0 - Q_0 u - U_0 q - q u - \\
       & g (c\theta_0 c\phi_0  - c\theta_0  s\phi_0 \phi - s\theta_0 c\phi_0 \theta)]
\end{split}
\end{equation*}

\noindent Subtracting $Z_0$ from both sides gives
\begin{equation*}
dZ = m[\dot{w} + P_0 v + V_0 p + p v - Q_0 u - U_0 q - q u - g (- c\theta_0  s\phi_0 \phi - s\theta_0 c\phi_0 \theta)]
\end{equation*}

\noindent Simplifying some signs and removing products of disturbance quantities
\begin{equation*}
dZ = m[\dot{w} + V_0 p + P_0 v - U_0 q - Q_0 u + (g c\theta_0  s\phi_0) \phi + (g s\theta_0 c\phi_0) \theta]
\end{equation*}

\end{document}
